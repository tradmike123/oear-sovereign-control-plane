\documentclass[11pt]{letter}
\usepackage[utf8]{inputenc}
\usepackage{geometry}
\geometry{a4paper, margin=1in}

\address{Miguel Garay \\ Independent Researcher \\ \texttt{Elbichi123@gmail.com}}
\signature{Miguel Garay}

\begin{document}

\begin{letter}{arXiv Administration / Editorial Board \\ Subject: Submission of OEAR Technical Architecture (cs.SE)}

\opening{To the Editorial Board,}

I am pleased to submit the original research paper titled \textbf{"OEAR: A Deterministic Sovereign Control Plane for Verifiable Governance of Probabilistic Cognitive Systems"} for consideration in the \textbf{Software Engineering (cs.SE)} category.

As Large Language Models (LLMs) and probabilistic inference engines become integrated into critical infrastructure, the software engineering community faces a unique challenge: managing non-deterministic outputs within deterministic software lifecycles. Traditional moderation and post-hoc filtering methods lack the rigor required for verifiable safety and auditability.

Our work, \textbf{OEAR}, addresses this gap by introducing a deterministic sovereign control plane. The key contributions of this paper that align with the highest standards of software engineering research include:

\begin{enumerate}
    \item \textbf{Mechanical Certification}: A methodology for establishing "Mechanical Truth" through versioned synthetic harnesses, replacing subjective trust with reproducible evidence.
    \item \textbf{Forensic Integrity}: An append-only journaling system that enforces temporal monotonicity and causal correlation, providing a verifiable log for post-incident analysis.
    \item \textbf{Longitudinal Governance}: A novel approach to tracking governance "grip" and drift across software versions using automated telemetry and shadow auditing.
    \item \textbf{Artifact Transparency}: The inclusion of a cryptographically congealed baseline vector set (SHA-256) that allows for independent verification of the governance thresholds described in the text.
\end{enumerate}

The OEAR architecture has been implemented and tested through an automated pipeline that integrates governance certification directly into Git Quality Gates. We have included a full reproducibility section with commands to verify the artifact's integrity on any local system.

We believe OEAR provides a foundational framework for the next generation of responsible AI deployments, shifting the focus from model-centric discipline to system-centric mechanical verification.

Thank you for your time and consideration.

\closing{Sincerely,}

\end{letter}
\end{document}
